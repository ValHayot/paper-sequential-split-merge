\documentclass[10pt, conference, compsocconf]{IEEEtran}

% packages
\usepackage{algorithm}
\usepackage{algorithmic}
\usepackage{color}
\usepackage[pdftex]{graphicx}
\usepackage{hyperref}

% new commands
\newcommand{\todo}[1]{
  \color{red}\emph{[#1]}
  \color{black}
}


\begin{document}

\title{Sequential algorithms to split and merge large 3D images}

\maketitle

\begin{abstract}
  \todo{The abstract goes here. DO NOT USE SPECIAL CHARACTERS,
    SYMBOLS, OR MATH IN YOUR TITLE OR ABSTRACT.}  Authors (final order
  to be discussed when the paper shapes up): Yongping/Val\'erie,
  Yuhong Yan, Tristan. Perhaps Pierre Bellec.
\end{abstract}


\section{Introduction}

Three-dimensional images that cannot be loaded in memory are
increasingly found in a variety of disciplines. BigBrain, for
instance, is a 3D histological image of the human brain that
represents 1~TB of data organized in 3600 planes at full resolution
(1$\mu$m in-plane), and 76~GB at a 40$\mu$m isotropic resolution
commonly used in neurosciences~\cite{amunts2013bigbrain}. Other
examples found in medical imaging, our primary domain of interest,
include \todo{electromicroscopy,spectroscopy-francoise-peyrin}. As
such images would typically be processed on a computing cluster,
possibly using locality-aware file systems such as HDFS, software
libraries are needed to split and merged such images efficiently. We
introduce and compare a family of algorithms to address this problem
in a sequential environment made of a single storage disk. We are
aware of the need for parallel algorithms that would split and merge
images to and from an array of disks but we plan to describe such
algorithms in a subsequent paper based on the results of the present
one.

A disk is modeled by its read and write rates ($\alpha_r$ and
$\alpha_w$, in MB/s), and its seek speed ($\beta$, also in
MB/s).

Images are split into chunks representing 3D blocks or 2D slices. We
assume that a block or slice fits in memory. The BigBrain would
perhaps be split into 125 chunks of 600~MB. The decision to split an
image into slices or blocks, and the size of the resulting chunks, may
be constrained by the application. Some applications, for instance
spatial filtering, would commonly require blocks while other ones such
as acquisition artifact removal would rather work on
slices. Flexibility is thus required in the splitting
scheme. Applications processing voxels individually, for instance
histogram computation or k-means clustering, could work on either
slices or blocks.

The main problem encountered in the context of this paper is the fact
that, if not ordered properly, data reads and writes might result in
extensive seek times that, as we illustrate below, drastically limit
the I/O performance. This problem is obviously related to the ordering
of data bytes in image files, which we assume is arbitrary but known
to the algorithm. The problem is also related to the geometry of the
chunks to extract or merge, which we again assume arbitrary but known
to the algorithm. In a nutshell, we are seeking for algorithms to
reduce seek times while allowing for arbitrary chunk geometries and
arbitrary byte orders in data files. The main idea of the variations
presented in the remainder of the paper is to convert data orderings
in memory before writing them on disk. 

The literature looks remarkably scarce on the problem described
above. Parallel processing of 3D images has obviously been extensively
studied, but methods have focused more on geometrical approaches to
partition images and on load-balancing and task scheduling techniques
while we aim at algorithms to efficiently split or merge images
regardless of the geometry of the chunks \todo{do a serious literature
review on this topic}. Perhaps this lack of related works could be
attributed to the fact that disk I/O weren't a bottleneck until image
resolution reached a certain threshold. A relevant reference is the
description of the Open Connectome Data Cluster~\cite{burns2013open},
a data warehouse system that allows users to retrieve specific chunks
of large datasets. In this work, the problem of long seek times is
addressed by designing a specific file format based on space-filling
curves, that elegantly preserves spatial proximity on disk. However,
addressing the problem using a specific data organization optimizes
the system for specific chunk geometries, volumetric in this case,
while we aim at algorithms that support multiple types of geometries.


\section{Algorithms}

In our context, split and merge relate to the same dual
problem \todo{not so sure about this actually}. Without loss of
generality, we focus here on merging for the sake of concision. Our
goal then is to merge a set of $N$ chunks into a single reconstructed
3D image with $F$ voxels of size $u$. We assume that all chunks are of
identical size and that slices are squares and blocks are cubes.

Without loss of generality, we consider a file format where voxels are
written after alphanumerical sorting of their (x,y,z)
representation. That is, x slices are written one after the other, and
in every x slice, y rows are written one after the other. NifTI, the
file format defined by the Neuroimaging Informatics Technology
Initiative Nifti\footnote{\url{https://nifti.nimh.nih.gov}} is such a
format, used by the vast majority of the neuroimaging community. In
the remainder, we use the term "slice" for "x slice".


\subsection{Slices vs blocks}

Algorithms~\ref{algo:naive-slices} and~\ref{algo:naive-blocks} are the
naive solutions to reconstruct large images from slices or blocks.
\begin{algorithm}{tbh}
\caption{Naive merging of slices.}
\label{algo:naive-slices} 
\begin{algorithmic}
  \FOR{each slice}
    \STATE read slice
    \STATE write slice in reconstructed image
  \ENDFOR      
\end{algorithmic}
\end{algorithm}

\begin{algorithm}{tbh}
\caption{Naive merging of blocks.}
\label{algo:naive-blocks}
\begin{algorithmic}
  \FOR{each block}
    \STATE read block
    \STATE write block in reconstructed image
  \ENDFOR 
\end{algorithmic}
\end{algorithm}

Although simplistic, Algorithms~\ref{algo:naive-slices}
and~\ref{algo:naive-blocks} actually exhibit very different
performance due to the amount of seeks performed by Algorithm 1.b. The
number of seeks performed in Algo 1.b for each block is indeed the
number of rows in a slice, that is, $(F/N)^{1/3}$. The total number of
seeks, considering $N$ blocks is thus $N^{2/3}.F^{1/3}$. In practice,
this could lead to a tremendous slowdown.

\subsection{Sorted blocks}

Sorting blocks alphanumerically by (x,y,z) as shown on Algorithm 2 reduces the seek time
between blocks.

Algo 2: reconstruction from sorted blocks.
* Sort blocks according to (x,y,z)
* For each block in sorted list:
  - read block
  - write block in reconstructed image

The number of seeks is not reduced but the length of seeks is.

\subsection{Cluster reads}

We fill memory with as many blocks as possible to write in slices. A
given block is read only once. Slices may be written partially.

Algo 3: reconstruction from blocks using cluster reads and sorted blocks
* Sort blocks according to (x,y,z)
* While there are unread blocks:
  * While there is memory available:
    - read next block in sorted list
  * (Partially) reconstruct slices from blocks (in memory).
  * For each partially reconstructed slice:
    - Write in reconstructed image

\subsection{Multiple reads}

We fill memory with as many voxels as possible to write groups of
entire slices. A given block may be read multiple times.

Algo 4: reconstruction from blocks using multiple reads
* While there are empty slices in the reconstructed image:
  * Determine $n$ so that $n$ slices fit in memory
  * Reconstruct next slices $s_1$, ...,  $s_n$:
     - list the blocks contributing to $s_1$, $s_n$
     - for each block in the list, read the voxels of interest
     - reconstruct $s_1$, ..., $s_n$ in memory
     - write $s_1$, ..., $s_n$

\subsection{Lossless compression}
     
Lossless compression might reduce overall read and write times by
reducing seek time and data sizes. However, it needs to be implemented
on-the-fly, i.e., in chunks. See the work in~\cite{rajna2015speeding}
and \href{https://github.com/pauldmccarthy/indexed\_gzip}{corresponding
  implementation}.
     
\subsection{Implementation and Benchmark}



As a running example to benchmark our algorithms, we use the 76~GB
BigBrain image split in 125 chunks. The data is available [here]. We
perform experiments using the two disks below:
- HD ...
- SSD ...
Our algorithms are implemented in Python using the Nibabel library. 

\section{Results}

-- Splitting, 125 blocks, HD vs SSD.
Figure 1: on HD, 5 algorithms, 10 repetitions per algorithm.
Figure 2: on SSD, 5 algorithms, 10 repetitions per algorithm. 

-- Merging, 125 blocks, HD vs SSD.
Figure 3: on HD, 5 algorithms, 10 repetitions per algorithm.
Figure 4: on SSD, 5 algorithms, 10 repetitions per algorithm. 

-- HD, 250 blocks vs 125 blocks. 
Figure 5: different block sizes

-- HD, 125 blocks, F vs F/2
- Figure 6: Different images sizes, with constant memory.

\section{Discussion}

What is the best algorithm (between 3 and 4), in which context. Did we
solve the seek problem or is there any issue remaining?

File formats:
\begin{itemize}
\item would MINC~\cite{vincent2016minc} support such algorithms? See on-going email discussion
with P. Bellec.
\item some formats optimize the storage for particular split shapes. See
  ndstore~\cite{burns2013open}. Such formats would behave equally
  poorly with splits that do not comply to this geometry. Our
  algorithms would work equally well with such formats.
\end{itemize}

Reconstructing (therefore splitting) a complete 3D image may not
always be necessary. Re-splitting, for instance, could be done from
existing splits.

\section{Conclusion}

Future work:
- obliques
- parallel algorithms
- re-splitting
- model

\section*{Acknowledgment}

(If not added as co-authors): Greg Kiar, Pierre Bellec.

\bibliographystyle{IEEEtran}
\bibliography{IEEEabrv,biblio.bib}

\end{document}
